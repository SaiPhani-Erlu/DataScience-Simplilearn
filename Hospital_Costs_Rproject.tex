#Title  : "Healthcare Cost"
#Author : "sai Phaneendra Kumar"

#---------------------------------------------------------------------------
getwd()
setwd("E:\\Learning\\SimpliLearn\\DataScientist\\Data Science with R Programming\\Projects\\HealthCare")

# Importing data sets
hosp<-read.csv("E:\\Learning\\SimpliLearn\\DataScientist\\Data Science with R Programming\\Projects\\HealthCare\\hospitalcosts.csv",header = T)
head(hosp) # Analysis: 500 observations and 6 Variables

summary(hosp)

attach(hosp)
# Goal/Expectation (1): To record the patient statistics, the agency wants to find the age category of people 
# who frequent the hospital and has the maximum expenditure

hist(AGE)

# To see the value of category of infants
ag <- as.factor(AGE)
summary(ag)

# Age category of 0 seems to be  frequently using the hospital
tapply(TOTCHG,AGE,sum)


which.max(tapply(TOTCHG,AGE,sum))

# Analysis(1): Max expenditure also by infant of 0 age =678118, 15=111747 17=174777

# Goal/Expectation(2): In order of severity of the diagnosis and treatments and to find out the expensive treatments, 
# the agency wants to find the diagnosis related group that has maximum hospitalization and expenditure

diagg <- as.factor(APRDRG)
summary(diagg)

which.max(summary(diagg))

tapply(TOTCHG,diagg,sum)

which.max(tapply(TOTCHG,diagg,sum))

max(tapply(TOTCHG,diagg,sum))

#Analysis(2): From the results we can see that the category 640 has the maximum entries of hospitalization
#and also has the highest total hospitalization cost (437978).   

# Goal/Expectation(3): To make sure that there is no malpractice, the agency needs to analyse
# if the race of the patient is related to the hospitalization costs

# h0:The race of the patient is related to the hospitalization costs. 
# ha:no relation

rc <- as.factor(RACE)
summary(rc)

# Now to omit na values from data set
hospna <- na.omit(hosp)
modelannova <- aov(TOTCHG~RACE)
summary(modelannova)

# Analysis(3): p-value comes out to be very high 68% this means we can take risk and reject the null hypothesis
# this means  there is no relation between the race of patient and the hospital cost. 

# Goal/Expectation(4) : To properly utilize the costs, the agency has to analyse 
# the severity of the hospital costs by age and gender for proper allocation of resources

model1 <- lm(TOTCHG~AGE+FEMALE)
summary(model1)


# Analysis(4):  p-value for age is very less this means it is a  important factor in the hospital costs as seen by the significance levels and p-values
# gender has also less p value means it is also having the impact on cost and same with intercept

# Goal/Expectation(5): Since the length of stay is the crucial factor for inpatients, 
# the agency wants to find if the length of stay can be predicted from age, gender, and race

model2 <- lm(LOS~AGE+FEMALE+RACE)
summary(model2)


# Analysis(5): Except for the intercept.
# The very high p-value signifies that there is no linear relationship between the given variables.
# That is, with just the age, gender, and race, it is not possible to predict the los of a patient


# Goal/Expectation(6) : To perform a complete analysis, the agency wants to find the variable
# that mainly affects the hospital costs

model3 <- lm(TOTCHG~ .,data=hospna) 
summary(model3)

# Analysis(6) : APRDRG also affect
# We can see that age and length of stay affect the total hospital cost.